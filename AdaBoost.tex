\documentclass{article}
\usepackage{xeCJK}
\usepackage{geometry}
\setCJKmainfont{SimSun}
\geometry{left=2cm,right=2cm,top=2.5cm,bottom=2.5cm}

\begin{document}
\title{AdaBoost元算法}
\author{xuhao}
\maketitle

\setcounter{page}{1}
\setlength{\baselineskip}{14pt}

\section{AdaBoost算法原理}
AdaBoost的自适应在于:前一个分类器分错的样本会被用来训练下一个分类器。
AdaBoost是一种迭代算法,在每一轮中加入一个新的弱分类器,直到达到某个预定的足够小的错误率。
每一个训练样本都被赋予一个权重,表明它被某个分类器选入训练集的概率。
如果某个样本点已经被准确地分类,那么在构造下一个训练集中,它被选中的概率就被降低;
相反,如果某个样本点没有被准确地分类,那么它的权重就得到提高。


最初令每个样本的权重都相等,对于第k次迭代操作,根据这些权重来选取样本点,进而训练分类器。
根据这个分类器,来提高被它分错的的样本的权重,并降低被正确分类的样本权重。
然后,权重更新过的样本集被用于训练下一个分类器。
\end{document}
